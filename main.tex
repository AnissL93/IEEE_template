\documentclass[conference]{IEEEtran}
\usepackage{amsmath,amsfonts}
\usepackage{algorithmic}
\usepackage{array}
\usepackage[table]{xcolor}
\usepackage[caption=false,font=normalsize,labelfont=sf,textfont=sf]{subfig}
\usepackage{textcomp}
\usepackage{stfloats}
\usepackage{url}
\usepackage{verbatim}
\usepackage{graphicx}
\usepackage{booktabs}
\usepackage{multirow}
\usepackage[normalem]{ulem}
\hyphenation{op-tical net-works semi-conduc-tor IEEE-Xplore}
\def\BibTeX{{\rm B\kern-.05em{\sc i\kern-.025em b}\kern-.08em
    T\kern-.1667em\lower.7ex\hbox{E}\kern-.125emX}}
\usepackage{balance}
\usepackage{caption}
% \usepackage{subcaption}
\usepackage{hhline}
\usepackage{pgfplotstable}
\usepackage{pgfplots}
\usepackage{xcolor}
\usepgfplotslibrary{groupplots} % Import the group plots library
\captionsetup{skip=4pt}
\usepackage{listings}
\usepackage{algorithmic}
\usepackage{algorithm}
% \usepackage{algpseudocode}
\usepackage{blindtext}
\usepackage{etoolbox}
\newcommand{\equal}[2]{\ifstrequal{#1}{#2}{true}{false}}

\usepackage{ifthen}

% uncomment this to generate submission version while using overleaf
% \newcommand{\submitmode}{true}


\ifthenelse{\equal{\submitmode}{true}}{
 %%submission mode
 \newcommand{\rmk}[1]{}
 \newcommand{\rmkdone}[1]{}
 \newcommand{\del}[1]{}
 \newcommand{\add}[1]{#1}
 \newcommand{\note}[1]{#1}
 \newcommand{\main}[1]{#1}
 \newcommand{\para}[2]{#1#2}
 \newcommand{\modify}[1]{}
}
{
\newcommand{\rmk}[1]{\textcolor{orange}{--[#1]--}}%%communication and remarks
\newcommand{\del}[1]{\textcolor{gray}{\sout{#1}}}%%delete the text
\newcommand{\rmkdone}[1]{\textcolor{gray}{--[#1]--}}%%done for remarks
\newcommand{\add}[1]{\textcolor{teal}{#1}}%%add
\newcommand{\note}[1]{\textcolor{red}{#1}}
\newcommand{\main}[1]{\textcolor{teal}{Main idea: [#1]}}
\newcommand{\para}[2]{\textcolor{gray}{Para #1: #2}}
\newcommand{\modify}[1]{\textcolor{blue}{#1}}
}


% correct bad hyphenation here
\hyphenation{op-tical net-works semi-conduc-tor}


\begin{document}
%
% paper title
% Titles are generally capitalized except for words such as a, an, and, as,
% at, but, by, for, in, nor, of, on, or, the, to and up, which are usually
% not capitalized unless they are the first or last word of the title.
% Linebreaks \\ can be used within to get better formatting as desired.
% Do not put math or special symbols in the title.
\title{Title for the paper}


% author names and affiliations
% use a multiple column layout for up to three different
% affiliations
% \author{\IEEEauthorblockN{Michael Shell}
% \IEEEauthorblockA{School of Electrical and\\Computer Engineering\\
% Georgia Institute of Technology\\
% Atlanta, Georgia 30332--0250\\
% Email: http://www.michaelshell.org/contact.html}
% \and
% \IEEEauthorblockN{Homer Simpson}
% \IEEEauthorblockA{Twentieth Century Fox\\
% Springfield, USA\\
% Email: homer@thesimpsons.com}
% \and
% \IEEEauthorblockN{James Kirk\\ and Montgomery Scott}
% \IEEEauthorblockA{Starfleet Academy\\
% San Francisco, California 96678--2391\\
% Telephone: (800) 555--1212\\
% Fax: (888) 555--1212}}

% conference papers do not typically use \thanks and this command
% is locked out in conference mode. If really needed, such as for
% the acknowledgment of grants, issue a \IEEEoverridecommandlockouts
% after \documentclass

% for over three affiliations, or if they all won't fit within the width
% of the page, use this alternative format:
% 
%\author{\IEEEauthorblockN{Michael Shell\IEEEauthorrefmark{1},
%Homer Simpson\IEEEauthorrefmark{2},
%James Kirk\IEEEauthorrefmark{3}, 
%Montgomery Scott\IEEEauthorrefmark{3} and
%Eldon Tyrell\IEEEauthorrefmark{4}}
%\IEEEauthorblockA{\IEEEauthorrefmark{1}School of Electrical and Computer Engineering\\
%Georgia Institute of Technology,
%Atlanta, Georgia 30332--0250\\ Email: see http://www.michaelshell.org/contact.html}
%\IEEEauthorblockA{\IEEEauthorrefmark{2}Twentieth Century Fox, Springfield, USA\\
%Email: homer@thesimpsons.com}
%\IEEEauthorblockA{\IEEEauthorrefmark{3}Starfleet Academy, San Francisco, California 96678-2391\\
%Telephone: (800) 555--1212, Fax: (888) 555--1212}
%\IEEEauthorblockA{\IEEEauthorrefmark{4}Tyrell Inc., 123 Replicant Street, Los Angeles, California 90210--4321}}




% use for special paper notices
%\IEEEspecialpapernotice{(Invited Paper)}




% make the title area
\maketitle

% As a general rule, do not put math, special symbols or citations
% in the abstract
\begin{abstract}
The abstract goes here.
\end{abstract}

% no keywords

% \begin{IEEEkeywords}
% component, formatting, style, styling, insert
% \end{IEEEkeywords}


% For peer review papers, you can put extra information on the cover
% page as needed:
% \ifCLASSOPTIONpeerreview
% \begin{center} \bfseries EDICS Category: 3-BBND \end{center}
% \fi
%
% For peerreview papers, this IEEEtran command inserts a page break and
% creates the second title. It will be ignored for other modes.
\IEEEpeerreviewmaketitle


%%%%%%%%%%%%%%%%%%%% Start context here %%%%%%%%%%%%%%%%%%%%

\section{Introduction}

Introduction goes here.

Example of citing~\cite{291164}.


\note{note}

\rmk{rmk}

\rmkdone{rmkdone}

\del{del}

\add{add}

\main{main}

\para{1}{para}

\modify{modify}

\section{Method}

\section{Experiments}

\section{Related Works}



\section{Conclusion}
The conclusion goes here.



% use section* for acknowledgment
% \section*{Acknowledgment}


% The authors would like to thank...


\bibliographystyle{IEEEtran}
\bibliography{ref}


\end{document}
